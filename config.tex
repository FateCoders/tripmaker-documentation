% --- CONFIGURAÇÕES E PACOTES ---

% Pacotes fundamentais
\usepackage{lmodern}
\usepackage[T1]{fontenc}
\usepackage[utf8]{inputenc}
\usepackage{indentfirst}
\usepackage{xcolor}
\usepackage{graphicx}
\usepackage{microtype}

% Pacotes adicionais
\usepackage{multicol}
\usepackage{multirow}
\usepackage{lipsum}

% Pacotes de citações
\usepackage[brazilian,hyperpageref]{backref}
\usepackage[alf]{abntex2cite}

% --- AJUSTE PARA REMOVER "Nenhuma citação no texto" ---
\renewcommand{\backrefpagesname}{Citado na(s) página(s):~}
\renewcommand{\backref}{}
\renewcommand*{\backrefalt}[4]{%
  \ifcase #1 
    Citado na página #2.%
  \else
    Citado #1 vezes nas páginas #2.%
  \fi
}

% Formatação para trechos de código
\usepackage{listings}
\lstdefinestyle{javaStyle}{
    language=Java,
    basicstyle=\ttfamily\small,
    keywordstyle=\color{blue!80!black}\bfseries,
    stringstyle=\color{red!70!black},
    commentstyle=\color{green!50!black},
    numberstyle=\scriptsize\color{gray!80},
    identifierstyle=\color{black},
    backgroundcolor=\color{gray!10},
    frame=shadowbox,
    framerule=0.5pt,
    rulecolor=\color{gray!50},
    breaklines=true,
    breakatwhitespace=true,
    breakindent=10pt,
    numbers=left,
    numbersep=10pt,
    showstringspaces=false,
    aboveskip=1em,
    belowskip=1em,
    xleftmargin=2em,
    xrightmargin=0.5em,
    emphstyle=\color{teal}\bfseries,
    morekeywords={public, class, void, new, implements, interface, private, final},
    literate={á}{{\'a}}1 {à}{{\`a}}1 {ã}{{\~a}}1 {â}{{\^a}}1 {ä}{{\"a}}1
             {é}{{\'e}}1 {è}{{\`e}}1 {ê}{{\^e}}1 {ë}{{\"e}}1
             {í}{{\'i}}1 {ì}{{\`i}}1 {î}{{\^i}}1 {ï}{{\"i}}1
             {ó}{{\'o}}1 {ò}{{\`o}}1 {õ}{{\~o}}1 {ô}{{\^o}}1 {ö}{{\"o}}1
             {ú}{{\'u}}1 {ù}{{\`u}}1 {û}{{\^u}}1 {ü}{{\"u}}1
             {ç}{{\c{c}}}1
             {Á}{{\'A}}1 {À}{{\`A}}1 {Ã}{{\~A}}1 {Â}{{\^A}}1 {Ä}{{\"A}}1
             {É}{{\'E}}1 {È}{{\`E}}1 {Ê}{{\^E}}1 {Ë}{{\"E}}1
             {Í}{{\'I}}1 {Ì}{{\`I}}1 {Î}{{\^I}}1 {Ï}{{\"I}}1
             {Ó}{{\'O}}1 {Ò}{{\`O}}1 {Õ}{{\~O}}1 {Ô}{{\^O}}1 {Ö}{{\"O}}1
             {Ú}{{\'U}}1 {Ù}{{\`U}}1 {Û}{{\^U}}1 {Ü}{{\"U}}1
             {Ç}{{\c{C}}}1
}
\lstset{style=javaStyle}

% ---
% Informações personalizadas
\newcommand{\alunoA}{João Victor de Barros Santos de Marins}
\newcommand{\alunoB}{Luiz Fiuza Barbosa de Freitas}
\newcommand{\alunoC}{Marlon Victor Bezerra dos Passos}
\newcommand{\alunoD}{Pedro de Sousa Vicente Menck}
\newcommand{\alunoE}{Vitor Henrique Fantes}

\newcommand{\alunos}{\alunoA \\ \alunoB \\ \alunoC \\ \alunoD \\ \alunoE}
\newcommand{\professor}{Caio Sobrenome}
\newcommand{\disciplina}{Projeto Integrador II}
\newcommand{\tema}{Turismo no Interior do Estado de São Paulo}
\newcommand{\nomeinstituicao}{Fatec Tatuí - Prof. Wilson Roberto Ribeiro de Camargo}
\newcommand{\areainstituicao}{Faculdade de Tecnologia de Tatuí}

% Informações de dados para CAPA e FOLHA DE ROSTO
\titulo{TRIPMAKER: Sistema de Roteirização Turística Inteligente no Interior do Estado de São Paulo}
\autor{\alunos}
\local{Tatuí/SP}
\data{2025, 2º Semestre}
\instituicao{
  \nomeinstituicao
  \par
  \areainstituicao
}
\tipotrabalho{Desenvolvimento de Aplicação}
\preambulo{Desenvolvimento de aplicação voltada ao \tema, com finalidade de obtenção de nota e presença para as aulas de \disciplina. \\ \\ Professor Orientador: \professor}

% Configurações de aparência do PDF
\definecolor{blue}{RGB}{41,5,195}
\makeatletter
\hypersetup{
    pdftitle={\@title},
    pdfauthor={\@author},
    pdfsubject={\imprimirpreambulo},
    pdfcreator={LaTeX with abnTeX2},
    pdfkeywords={abnt}{latex}{abntex}{abntex2}{relatório técnico},
    colorlinks=true,
    linkcolor=blue,
    citecolor=blue,
    filecolor=magenta,
    urlcolor=blue,
    bookmarksdepth=4
}
\makeatother

% Espaçamentos entre linhas e parágrafos
\setlength{\parindent}{1.3cm}
\setlength{\parskip}{0.2cm}

% Compila o índice
\makeindex
